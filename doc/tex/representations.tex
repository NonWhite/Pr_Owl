\section{Representações de ontologias probabilísticas}
\label{sec:resumo}

Aquela representação precisada pode ser encontrada extendendo o conceito de ontologias a ontologias probabilísticas. As ontologias probabilísticas estendem o conceito das ontologias comuns para lidar com incerteza sobre conceitos ambíguos, não confiáveis ou incompletos, sejam estos relacionados com tipos, propriedades ou relações. Existem varias abordagens feitas para encontrar um jeito de representar um modelo de conhecimento baseado na lógica que lide com incerteza ao mesmo tempo. Entre as principais estão Hidden Markov Models (HMM), Bayesian Networks (BNs), Probabilistic Relational Models (PRM), entre outras~\cite{Costa10}, mas todas elas tem desvantagens ou não conseguem combinar ambas coisas satisfatoriamente.

Mas no ano 2008, Laskey conseguiu desenvolver a primeira representação que combina lógica com incerteza de forma satisfatória. Aquela representação está baseada em um tipo de rede Bayesiana chamado Multi-Entity Bayesian Network (MEBN) onde cada sentença em lógica de primeira ordem pode ser representada como um fragmento ou MFrag. Cada um desses fragmentos tem distribuições de probabilidades locais e o conjunto de MFrags construi toda a base de conhecimento~\cite{Laskey08}.