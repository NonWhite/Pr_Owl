\section{Definições}
\label{sec:definitions}

\subsection{Ontologias}
\label{subsec:ontolgy}

No ano 2004, World Wide Web Consortium estabeleceu como padrão a linguagem OWL (Web Ontology Language) para representar ontologias que eram a principal representação para conseguir uma web semântica. Uma ontologia é uma representação formal de conhecimento baseada na lógica de descrição que inclui tipos de entidades (e.g. Pessoa, Companhia), propriedades de aquelas entidades (e.g. nome, sobrenome), relações entre entidades (e.g. paiDe) e eventos que acontecem com aquelas entidades. Por ser baseada em lógica, existe a possibilidade de fazer inferências sobre os fatos estabelecidos. Além disso, também pode ser usada para ajudar os motores de busca responder a perguntas mais complexas.

O principal problema de usar as ontologias comuns e OWL como meios para obter a web semântica é que a web atual tem incerteza em muitos conceitos. A incerteza poderia estar não só em tipos de entidades (e.g. Python é uma linguagem de programação ou um animal), se não também em propriedades ou relações que tem diferentes comportamentos em diferentes domínios (e.g. limpo pode significar que não tem registros policiais, ou que não está sujo). Tendo isto em consideração, existe uma necessidade de encontrar uma representação que use tanto lógica como teoria de probabilidades para representar o conhecimento e também de extender a linguagem OWL para ter suporte de probabilidades.

\subsection{Ontologias Probabilísticas}
\label{subsec:prob_ontology}