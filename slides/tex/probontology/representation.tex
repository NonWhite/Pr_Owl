\subsection{Representation}

\begin{frame}
	\centering{\Large{How to represent \alert{probabilistic} ontologies?}}
	\pause
	\begin{block}{Some possible solutions}
		\begin{itemize}
			\item Plates
			\begin{itemize}
				\item Represent fragments of graphical models
				\item Very useful with continuous attribute values
				\item Can not represent directly uncertainty
			\end{itemize}
			\pause
			\item Bayesian Networks (BNs)
			\begin{itemize}
				\item Represent uncertainty
				\item Limited to attribute-value representation
			\end{itemize}
			\pause
			\item Hidden Markov Models
			\begin{itemize}
				\item They are a special case of Dyanmic BNs
				\item Capability for recursion
			\end{itemize}
		\end{itemize}
	\end{block}
\end{frame}

\begin{frame}
	\centering{\Large{How to represent \alert{probabilistic} ontologies?}}
	\begin{block}{Some possible solutions}
		\begin{itemize}
			\item Probabilistic Relational Models
			\begin{itemize}
				\item Extend BNs to handle multiple entity types
				\item Can not express quantified first-order sentences
				\item Does not support recursion
			\end{itemize}
			\pause
			\item DAPER
			\begin{itemize}
				\item Combine entity-relational model with DAGs models
				\item Express probabilistic knowledge
				\item Does not support quantifiers
			\end{itemize}
		\end{itemize}
	\end{block}
	All of these representations can not successfully combine \alert{Logic} and \alert{Uncertainty}
\end{frame}